\documentclass[12pt]{book}

\usepackage[]{amsmath}
\usepackage[]{amsthm}
\usepackage[]{amsfonts}
\usepackage[]{amssymb}
\usepackage{blindtext}
\usepackage[a4paper, total={6in, 8in}]{geometry}

\usepackage{listings}
\usepackage{color}

\definecolor{dkgreen}{rgb}{0,0.6,0}
\definecolor{gray}{rgb}{0.5,0.5,0.5}
\definecolor{mauve}{rgb}{0.58,0,0.82}

\lstset{frame=tb,
  language=Java,
  aboveskip=3mm,
  belowskip=3mm,
  showstringspaces=false,
  columns=flexible,
  basicstyle={\small\ttfamily},
  numbers=left,
  numberstyle=\small\color{black},
  keywordstyle=\color{blue},
  commentstyle=\color{dkgreen},
  stringstyle=\color{mauve},
  breaklines=true,
  breakatwhitespace=true,
  tabsize=4
}

\title{EECS3101 notes::Dynamic programming examples}

\author{Jerry Wu}
\date{2023-02-28}

\begin{document}

\maketitle

\chapter*{Week 7 practice}

\subsection*{Example 1:: Optimizing an itinerary}

The details of the problem are as follows::

\begin{itemize}
    \item We wish to travel from city 0 to city $n$ using buses.
    \item The cost of going from $i$ to $j$ is defined as $C_{i,j}, \forall i,j(i<j)$.
    \item All buses only go from a lower numbered city to a higher numbered city.
    \item What is the minimum cost of going from 0 to n.
\end{itemize}

\subsection*{Solution}
\begin{quote}
    \textit{"If you have an insecurity about your final answer, you might have some reviewing to do."-Larry YL Zhang 2023}
\end{quote}
\begin{itemize}
    \item[\textbf{1.}] List out the subproblems\\
    What is the minimum cost of going from $i$ to $j$? What about $i$ to $n$ or $0$ to $i$? We can try the last one because the main problem is from $0$ to $n$. Let $ans[i]=minCost(0\rightarrow i)$
    \item[\textbf{2.}] Find a recursive relation between the subproblem solutions. Consider an intermediate value between $0$ and $j$ defined as $k\in[0,j-1]\equiv k\in[0,j)$ which is a node directly \textbf{preceding} j.
    \begin{align*}
        ans[j]=\min_{\forall k\in[0,j)}\{ans[k]+C_{k,j}\}
    \end{align*}
    With this, we're able to find the minimum of all the minimums (the absolute minimum). This search is exhaustive.
    \begin{quote}
        \textit{"Don't overthink it. It should be straightforward."-Larry YL Zhang 2023}
    \end{quote}

    \item[\textbf{3.}] Data structure to store the answers?\\ We have a problem that is $O(n)=\Theta(n)$ because $j\in [0,n]$, so our runtime for the calculation is also $O(n)=\Theta(n)$. So the overall runtime would be $\Theta(n^2)$. Because we have $n$ cities, we can safely say that the space complexity is $\Theta(n)$.
    
    \subsection*{Base case}
    For the base case, we can just set $k=0$. It is the case where we don't move to any new city.
    \begin{align*}
        ans[0]=ans[0]+C_{0,0}=0
    \end{align*}
\end{itemize}

\subsection*{The code}

So now, we can code our solution.

\begin{lstlisting}
CheapestItinerary(C)
{
    ans = array of size n+1;
    ans[0] = 0;
    for(i = 1; i <= n; i++)
    {
        ans[i] = +infinity; //unvisited node, set to a large value
        for(k = 0; k < i; k++)
        {
            ans[i] = min(ans[i], ans[k] + C[k,i])
        }
    }

    return ans[n];
}

\end{lstlisting}
\newpage
\subsection*{Example 2:: Matrix chain multiplication}

Suppose we have two arbitrary matrices::

\begin{itemize}
    \item $A$ is an $n\times m$ matrix
    \item $B$ is an $m\times k$ matrix
    \item $C=AB$ which is an $n\times k$ matrix. produced via $n\times m\times k$ scalar multiplications.
\end{itemize}

\begin{quote}
    \textit{"Quick maths"-Larry YL Zhang 2023}
\end{quote}

How many scalar multiplications are needed with different parenthisization? Compute $A_1A_2A_3\ldots A_n$ with the \textbf{fewest} number of multiplications. \\

The size of $A_1$ is $d_0\times d_1$, $A_2$ is $d_1\times d_2$, $A_n$ is size $d_{n-1}\times d_n$. So the input is an array of $d_0$ to $d_n$ where size is $n+1$.

\subsection*{Solution}

We need to store the subproblems in a 2d array. How should we populate this 2d array to get the optimal solution? Well, we need to make sure that RHS is available before we can evaluate LHS, i.e. $ans[i][j]$ with a smaller $|j-1|$ must be evaluated first. So we can populate the 2d array diagonally, where the base cases are all the indices directly down the diagonal: $((0,0), (1,1), (2,2), (3,3)...(n,n))$. Illustration is available on \textbf{slide 19}.
\newpage
\subsection*{The code}

\begin{lstlisting}
//input d is an array of size n+1
//we are storing the dimensions of the matrices in the array

int matrixMultiplication(int[] d)
{
    int[][] ans = a 2d array of size (n+1)x(n+1)
    for(i = 1; i <= n; i++)
    {
        ans[i][i] = 0 //populate each element in the diagonal to be 0 (the base cases)
    }

    //diff is the difference between i and j

    for(diff = 1; diff <= n; diff++)
    {
        for(i = 1; i <= n; i++)
        {
            j = i + diff;
            ans[i][j] = +infinity
            for(k = i; k < j; k++)
            {
                ans[i][j] = min(ans[i][j], ans[i][k] + ans[k+1][j] + d[i-1]*d[k]*d[j])
            }
        }
    }
    //the optimal solution will always be in the first row.
    return ans[1][n]
}
\end{lstlisting}

\end{document}