\documentclass[12pt]{book}

\usepackage[]{amsmath}
\usepackage[]{amsthm}
\usepackage[]{amsfonts}
\usepackage[]{amssymb}
\usepackage{blindtext}
\usepackage{pgfplots}
\usepackage[a4paper, total={6in, 8in}]{geometry}

\title{Chapter 6::Chemical equilibrium}

\author{Jerry Wu}

\date{\today}

\begin{document}
\maketitle

\chapter*{Gibbs energy}
\subsection*{Abstract}
We would like to determine the spontaneity of a reaction mixture to approach equilibrium, along with deriving the thermodynamic equlibrium constant $K_p$ and equilibrium concentraations of reactants and products in a mixture of reactive ideal gases. We define Gibbs energy ($\Delta G$), which expresses spontaneity in terms of properties of the system alone.\\

Suppose we have an arbitrary system $\sigma$ in thermal equilibrium with the surroundings. What determines the directions of a spontaneous change? Well, it would be the tendency of a system to minimize energy.

\begin{align*}
    -dq_{\sigma}=dq_{surr}\implies dS_{\Omega}+dS_{surr}\geq 0\implies dS_{\sigma}+\frac{dq_{surr}}{T}\geq 0
\end{align*}

\subsection*{Definition}
\begin{align*}
    \Delta G_{\sigma}=\Delta H_{\sigma}-T\Delta S_{\sigma}\implies dG_{\sigma}=dH_{\sigma}-TdS_{\sigma}
\end{align*}

If we graph this relation, we notice that $\Delta G$ is a function of $T$. $\Delta H$ would be $\Delta G_0$ and the slope would be $\Delta S$.

Gibbs energy holds the following properties::

\begin{itemize}
    \item $\Delta G<0\implies \sigma\equiv spont\implies exergonic$
    \item $\Delta G>0\implies \sigma\equiv \lnot spont\implies endergonic$
    \item Gibbs energy is an extensive property, and is a state function.
\end{itemize}

When a problem concerns Gibbs energy, we only care about \textbf{what happens in the system, not the surroundings}.

\subsection*{Differential forms of $U$, $H$, and $G$}
For an infinitessimal process::
\begin{align*}
    G=H+TS\implies dG=dH-TdS-SdT
\end{align*}

According to the first law::
\begin{align*}
    dH=dU+PdV+VdP\implies dU=dq+dw\implies dU=dq-PdV
\end{align*}

For a reversible process::
\begin{align*}
    dq_{rev}=TdS, dU=TdS-PdV
\end{align*}

So,

\begin{align*}
    dH=TdS+VdP,dG=VdP-SdT, dU=TdS-PdV
\end{align*}

This only applies for \textbf{expansion} work.

\subsection*{Example}

Calculate the maximum nonexpansion work that can be gained from the combustion of
benzene $(l)$ and of $H_2$ (g) on a per gram and a per mole basis under standard conditions. Is it
apparent from this calculation why fuel cells based on hydrogen oxidation are under
development for mobile applications? Thermodynamic values available in appendix.

\begin{align*}
    R:C_6H_6(l)+O_2(g)\rightarrow 6CO_2(g)+3H_2O(l)
\end{align*}

\textbf{To calculate maximum nonexpansion work, simply calculate} $\Delta G$

\begin{align*}
    \Delta G=3\Delta G(H_2O,l)+6\Delta G(CO_2,g)-\Delta G(G_6H_6,l)\approx 3202.2kJ
\end{align*}

The above value is per mole, so we need to multiply by molar mass ($\frac{1mol}{78.18g}$). In the end, we get $-40.99 \frac{kJ}{g}$

For formation of water, we have that

\begin{align*}
    \Delta G=\Delta G(H_2O,l)=-237.1kJmol^{-1}\implies -237.1kJmol^{-1}\left(\frac{1mol}{2g}\right)=-117.6kJg^{-1}
\end{align*}

\subsection*{Helmholtz energy and spontaneity}

This property is derived for processes occuring at $V\equiv const \land T\equiv const$. We use $A$ to represent the quantity.

\begin{align*}
    A=U-TS
\end{align*}

Gibbs free energy gives \textbf{maximum nonexpansion work}, whereas Helmholtz energy gives \textbf{maximum overall work.}

\subsection*{Exact differentials and Maxwell relations}

Take a function $z=f(x,y)$ where $dz=M(x,y)dx+N(x,y)dy$

Recall that by Euler's theorem, a function is a state function if and only if::

\begin{align*}
    \left(\frac{\partial M}{\partial y}\right)_x=\left(\frac{\partial N}{\partial x}\right)_y
\end{align*}

The total differentials for $U(S,V), H(S,P), G(T,P)$ can be modelled as follows::

\begin{align*}
    dU=TdS-PdV=\frac{\partial U}{\partial S}dS+\frac{\partial U}{\partial V}dV|T=\frac{\partial U}{\partial S}, -P=\frac{\partial U}{\partial V}
\end{align*}

\begin{align*}
    dH=TdS+VdP=\frac{\partial H}{\partial S}dS+\frac{\partial H}{\partial P}dP|T=\frac{\partial H}{\partial S},V=\frac{\partial H}{\partial P}
\end{align*}

\begin{align*}
    dG=-SdT+VdP=\frac{\partial G}{\partial T}dT+\frac{\partial G}{\partial P}dP|-S=\frac{\partial G}{\partial T}, V=\frac{\partial G}{\partial P}
\end{align*}

Because $dU$ is an exact differential, 

\begin{align*}
    \frac{\partial}{\partial V}\left(\frac{\partial U(S,V)}{\partial S}\right)=\frac{\partial}{\partial S}\left(\frac{\partial U(S,V)}{\partial V}\right)
\end{align*}

It would follow that::

\begin{align*}
    \frac{\partial T}{\partial V}=-\frac{\partial V}{\partial S}
\end{align*}

\subsection*{Example}

Derive the maxwell relation $\left(\frac{\partial T}{\partial V}\right)_S=-\left(\frac{\partial P}{\partial S}\right)_V$

Because $S$ and $V$ are constant, we can use the exact differential::

\begin{align*}
    dU=TdS-PdV
\end{align*}

Now we can form the mixed derivative::

\begin{align*}
    \left(\frac{\partial}{\partial V}\left(\frac{\partial U(S,V)}{\partial S}\right)_V\right)_S=\left(\frac{\partial}{\partial S}\left(\frac{\partial U(S,V)}{\partial V}\right)_S\right)_V
\end{align*}

Substituting $dU=TdS-PdV$, we have that

\begin{align*}
    \left(\frac{\partial}{\partial V}\left(\frac{\partial\left[TdS-PdV\right]}{\partial S}\right)_V\right)_S=\left(\frac{\partial}{\partial S}\left(\frac{\partial\left[TdS-PdV\right]}{\partial V}\right)_S\right)_V
\end{align*}

After simplifying, we can conclude that

\begin{align*}
    \left(\frac{\partial T}{\partial V}\right)_S=-\left(\frac{\partial P}{\partial S}\right)_V\blacksquare
\end{align*}

\subsection*{Dependence of $\Delta G$ on $T$}

Recall that $dG=VdP-SdT$. At $P\equiv const,dG=-SdT$. When $P\equiv const$, it also follows that

\begin{align*}
    \frac{\partial G}{\partial T}=-S, S\geq 0
\end{align*}

So it would follow that Gibbs energy \textbf{decreases as energy increases}.

The Gibbs-Helmholtz equation states that for a finite process,

\begin{align*}
    \frac{\partial \left(\frac{\Delta G}{T}\right)}{T}=-\frac{\Delta H}{T^2}
\end{align*}

Assume that $\Delta H\land \Delta S$ are independent of T and that $P=1atm$. We can integrate both sides of the equation to give us that

\begin{align*}
    \int d\left(\frac{\Delta G}{T}\right)=\int\Delta Hd\frac{1}{T}
\end{align*}

\begin{align*}
    \frac{\Delta G(T)}{T}-\frac{\Delta G(298)}{298}\approxeq \left(\frac{1}{T}-\frac{1}{298}\right)\Delta H(298)
\end{align*}

So in the end, we have that the Gibbs-Helmholtz equation can be approximated as::

\begin{align*}
    \frac{\Delta G(T_2)}{T_2}-\frac{\Delta G(T_1)}{T_1}\approxeq \left(\frac{1}{T_2}-\frac{1}{T_1}\right)\Delta H(T_1)
\end{align*}

\subsection*{Dependence of $\Delta G$ on $P$}

Recall that $dG=VdP-SdT$, then at $T\equiv const$, we have that $dG=VdP\implies \frac{\partial G}{\partial P}=V$.\\

We model dependence of $\Delta G$ on $P$ as follows::

\begin{align*}
    \Delta G=\int_{1}^{2}dG=G_2-G-1=\int_{P_1}^{P_2}VdP
\end{align*}

If $V\equiv const$ and the process concerns liquids and solids, we have that::

\begin{align*}
    \Delta G=G(P_2)-G(P_1)=V(P_2-P_2)
\end{align*}

For $g\in \mathbb{C}$, we have that::

\begin{align*}
    \Delta G=G_2-G_1=\int_{P_1}^{P_2}\frac{nRT}{P}dP=nRT\ln\left(\frac{P_2}{P_1}\right)
\end{align*}

Now set $P_1=1bar$ and set $P_2=P$, therefore,

\begin{align*}
    G=G^{\circ}+nRTln(P)\implies \bar{G}=\bar{G^{\circ}}+RT\ln(P)
\end{align*}

If $\forall products,reactants \in \sigma((products\equiv ((s)\lor (l)))\lor (reactants \equiv((s)\lor (l))))$, then

\begin{align*}
    \Delta G=G(P_2)-G(P_1)=V(P_2-P_1)=V\Delta P
\end{align*}

This is not very important, since we can just ignore the quantity altogether.\\

However, if $\exists product\lor reactant\in \sigma(product\lor reactant\equiv (g))$, then the volume of solids and liquids is ignored, and::

\begin{align*}
    \Delta G=G(P_2)-G(P_1)=\Delta nRT\ln\left(\frac{P_2}{P_1}\right)
\end{align*}

\subsection*{Hess's law for Gibbs energy}
It's the same thing as enthalpy.

\subsection*{Example}

For the synthesis of urea::

\begin{align*}
    R:CO_2(g)+2NH_3(g)\rightarrow (NH_2)_2CO(s)+H_2O(l)
\end{align*}

\begin{itemize}
    \item[a)] Calculate $\Delta_r G$ for the reaction at $298K$ and $1 bar$ (Ans: $-6.8 kJ$)
    \item[b)] Assume $g\in \mathbb{C}$, calculate $\Delta_r G$ at $P=10bar$.
\end{itemize}

To calculate $\Delta G(10 bar)$, we can plug in the values into the equation we derived.

\begin{align*}
    \Delta G(10 bar)=\Delta G(1 bar)+\Delta nRT\ln\left(\frac{P_2}{P_1}\right)
\end{align*}

\begin{align*}
    =(6800Jmol^{-1})+(-3)(8.314JK^{-1}mol^{-1})(298K)(\ln\left(\frac{10}{1}\right)=-23.9kJmol^{-1})
\end{align*}

So it would suffice to say that a reaction at higher pressure is \textbf{more spontaneous} than at lower pressure.

\subsection*{Haber process}

\end{document}