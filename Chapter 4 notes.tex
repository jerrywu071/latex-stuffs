\documentclass[12pt]{book}

\usepackage[]{amsmath}
\usepackage[]{amsthm}
\usepackage[]{amsfonts}
\usepackage[]{amssymb}
\usepackage{blindtext}
\usepackage{pgfplots}
\usepackage[a4paper, total={6in, 8in}]{geometry}

\title{Chapter 4:: Thermochemistry}

\author{Jerry Wu (217545898)}

\date{\today}

\begin{document}
\maketitle
\chapter*{Thermochemistry}
\rule{\textwidth}{0.4pt}

\subsection*{Motivation}
We want to study the relation ship between energy and chemical reactions, along with transformations between forms of energy on a macroscopic scale. Recall that the following happens during reactions::

\begin{itemize}
    \item Bonds broken $\implies$ requires energy (endothermic, $\Delta H>0$)
    \item Bonds formed $\implies$ releases energy (exothermic, $\Delta H<0$)
    \item Temperature change
    \item Gases released or consumed (due to work)
\end{itemize}

\subsection*{Enthalpy of reactions}
\begin{itemize}
    \item $\forall R(P\equiv const)$, we have that $q_P=\Delta_r H$ (standard enthalpy of reaction).
    \item $\Delta_r H^{\circ}$ has units of kJ
    \item $\Delta_r H^{\circ}$ is the enthaly change when reactants in standard state are converted to products in \textbf{standard state}.
    \item Enthalpy is an extensive property.
\end{itemize}

\subsection*{Hess' law}

Recall that the standard change in enthalpy for a reaction can be calculated with::

\begin{align*}
    \Delta_r H^{\circ}=\sum_{products} v\Delta_f \bar{H}^{\circ}_{prod}-\sum_{reactants} v\Delta_f \bar{H}^{\circ}_{rxt}
\end{align*}

where $v$ is the stoichiometric coefficients of the products and reactants respectively.

\end{document}