\documentclass[12pt]{book}

\usepackage[]{amsmath}
\usepackage[]{amsthm}
\usepackage[]{amsfonts}
\usepackage[]{amssymb}
\usepackage{blindtext}
\usepackage{pgfplots}
\usepackage[a4paper, total={6in, 8in}]{geometry}

\title{Chapter 4:: Thermochemistry}

\author{Jerry Wu}

\date{\today}

\begin{document}
\maketitle
\chapter*{Thermochemistry}
\rule{\textwidth}{0.4pt}

\subsection*{Motivation}
We want to study the relationship between energy and chemical reactions, along with transformations between forms of energy on a macroscopic scale. Recall that the following happens during reactions::

\begin{itemize}
    \item Bonds broken $\implies$ requires energy (endothermic, $\Delta H>0$)
    \item Bonds formed $\implies$ releases energy (exothermic, $\Delta H<0$)
    \item Temperature change
    \item Gases released or consumed (due to work)
\end{itemize}

\subsection*{Enthalpy of reactions}
\begin{itemize}
    \item $\forall R(P\equiv const)$, we have that $q_P=\Delta_r H$ (standard enthalpy of reaction).
    \item $\Delta_r H^{\circ}$ has units of kJ
    \item $\Delta_r H^{\circ}$ is the enthaly change when reactants in standard state are converted to products in \textbf{standard state}.
    \item Enthalpy is an extensive property.
\end{itemize}

\subsection*{Hess' law}

Recall that the standard change in enthalpy for a reaction can be calculated with::

\begin{align*}
    \Delta_r H^{\circ}=\sum_{products} v\Delta_f \bar{H}^{\circ}_{prod}-\sum_{reactants} v\Delta_f \bar{H}^{\circ}_{rxt}
\end{align*}

where $v$ is the stoichiometric coefficients of the products and reactants respectively.

\subsection*{Standard enthalpies of formation}

The following property applies to elements ($\zeta$) in their standard state

\begin{align*}
    \forall \zeta \in \mathbb{S}(\Delta H_f^{\circ}(\zeta)=0)
\end{align*}

Enthalpy of formation always produces exactly \textbf{1 mol} of the species of interest. For example::

\begin{align*}
    R: C(s)_{graphite}+2H_2(g)\rightarrow CH_4(g)|\Delta H_f^{\circ}=-74.8kJ/mol
\end{align*}

In the above reaction, we assume $C(s)$ is graphite.

\begin{align*}
    \frac{3}{2}H_2(g)+\frac{1}{2}N_2(g)\rightarrow NH_3(g)|\Delta H_f^{\circ}=-46 kJ/mol
\end{align*}

Note that we \textbf{must} resrtict the reactant coefficient to 1 in order for it to be a formation reaction.

\subsection*{Standard state}

\begin{itemize}
    \item For a pure gas the pressure is exactlt $1bar$
    \item For liquids and solids, the pressure is 1 bar and the temperature is $25^{\circ}C$
    \item For a substance in solution, the concentration is exactly $1molL^{-1}$.
\end{itemize}

\subsection*{Example}

Calculate $\Delta H$ and $\Delta U$ at 298.15K for the reaction::

\begin{align*}
    R: 4NH_3(g)+6NO(g)\rightarrow 5N_2(g)+6H_2O(g)
\end{align*}

We have one pure element ($N_2(g)$), so it requires 0 energy by default.

So, we can plug them in

\begin{align*}
    \Delta_R H^{\circ}=\sum v\Delta H_f(prod)-\sum v\Delta H_f(rxt)
\end{align*}

\begin{align*}
    =5(0kJmol^{-1})-5(241.8kJmol^{-1})+4(45.9kJmol^{-1})-6(91.3kJmol^{-1})=-1815.0kJmol^{-1}
\end{align*}

Because $\Delta H<0$, we have that the reaction is exothermic. So we can now calculate our internal energy of the reaction.

\begin{align*}
    \Delta U(reaction)=\Delta H(reaction)-\Delta nRT
\end{align*}

\begin{align*}
    =(-1815.0)-1(8.314472jK^{-1}mol^{-1})(298.15K)=-1817.5kJmol^{-1}
\end{align*}

\subsection*{Obtaining $\Delta_f H^{\circ}$}

\begin{itemize}
    \item Direct method:: Only works for compounds that can be directly synthesized from their elements. This is not usually the case, so we have to use the indirect method.
    \item Indirect method:: based on Hess' law
    \begin{align*}
        \Delta H_{overall}=\sum_{i=1}^{n}\Delta H_i
    \end{align*}
\end{itemize}

\subsection*{Hess' law continued}

For a multi step process, we can sum up reactions to calculate different values of $\Delta H^{\circ}$. Take the Haber process as an example::

\begin{align*}
    R: 2H_2(g)+N_2(g)\rightarrow 2NH_3|\Delta H_3=-92.2kJ\\
    R_1: 2H_2(g)+N_2(g)\rightarrow N_2H_4(g)|\Delta H_1=?\\
    R_2:N_2H_4(g)+H_2(g)\rightarrow 2NH_3(g)|\Delta H_2=-187.6kJ
\end{align*}

We can calculate $\Delta H_1$ from subtracting $\Delta H_3$ from $\Delta H_2$. So we get $\Delta H_1=95.4kJ$.\\

When multiplying reactions by a factor, we also need to multiply their $\Delta H$ with the same factor.

\subsection*{Bond enthalpies}
For the dissocation of nitrogen and other gases, it has special meaning since there is only one bond.
\begin{align*}
    N_2(g)\rightarrow 2N(g)|\Delta_r H^{\circ}=941.4kJ
\end{align*}

This is really high because nitrogen gas is held together by a triple bond.\\

For polyatomic moecules, there's more than one bond, so we can refer to it as \textbf{average bond enthalpy}.

\begin{align*}
    H_2O(g)\rightarrow \frac{1}{2}H_2(g)+OH(g)|\Delta _r H^{\circ}=502kJ\\
    OH(g)\rightarrow O(g)+H(g)|\Delta_r H^{\circ}=427kJ
\end{align*}

$OH$ and $H_2O$ have the same elements, but the energy required to break them are different.

\subsection*{Example 2 (slide 19)}

Calculate the single bond enthalpy and energy for $Si-F$, given $\Delta H_f(SiF_4)=-1614.9 kJ/mol$

\begin{align*}
    R: Si(s)+2F_2(g)\rightarrow SiF_4(g)|\Delta H_f(SiF_4)=-1614.9 kJmol^{-1}\\
    R_1: SiF_4\rightarrow Si(s)+2F_2(g)|\Delta H_r=1614.9kJmol^{-1}\\
    R_2: 2F_2(g)\rightarrow 4F(g)|\Delta H_r=4(79.4)kJmol^{-1}\\
    R_3: Si(s)\rightarrow Si(g)|\Delta H_r=450kJmol^{-1}
\end{align*}

So the final reaction is::

\begin{align*}
    R_f: SiF_4(g)\rightarrow Si(g)+4F(g)|\Delta H_r=2382.5kJmol^{-1}
\end{align*}

So average $Si-F$ bond enthalpy can be calculated::

\begin{align*}
    E[\Delta H]=\frac{2382.5}{4}=596kJmol^{-1}
\end{align*}

And we can calculate internal energy.

\begin{align*}
    E[\Delta U]=\Delta H-\Delta nRT=2382.5-4(8.314)(298.15)=2372.6kJmol^{-1}
\end{align*}

Dividing $\Delta U$ by 4, we have that average $Si-F$ bond energy is $593kJmol^{-1}$

\subsection*{Dependence on temperature}

What happens when $T\neq 298.15K$?

\begin{align*}
    H_T^{\circ}=H_{298.15K}^{\circ}+\int_{298.15K}^{T}C_P(T')dT'
\end{align*}

So we can say that for a reaction where elements are included::

\begin{align*}
    \Delta C_P(T')=\sum_{i}v_i C_{P,i}(T')
\end{align*}

\begin{align*}
    \Delta_r H^{\circ}(T)=\Delta_r H^{\circ}(298.15K)+\int_{298.15K}^{T}\Delta C_P(T')dT' 
\end{align*}

\subsection*{Example 3 (slide 22)}
Calculate $\Delta H_f^{\circ}$ for $NO(g)$ at $840K$ assuming that the heat
capacities of reactants and products are constant over
the temperature interval at their values at $298.15K$. $\Delta H_f$ for NO(g) at $298.16K=91.3kJmol^{-1}$\\ We can calculate $\Delta C_P$ first.

\begin{align*}
    \Delta C_P=C_P(NO(g))-\frac{1}{2}C_P(N_2(g))-\frac{1}{2}C_P(O_2(g))=29.86-\frac{1}{2}(29.13)-\frac{1}{2}(29.38)=0.605 JK^{-1}mol^{-1}
\end{align*}

Full solution is on the slide.

\subsection*{Combustion}

The general form of a combustion of carbohydrates 
is::

\begin{align*}
    R_{comb}: C_x H_y+O_2(g)\rightarrow CO_2(g)+H_2O(l), x,y\in \mathbb{N}
\end{align*}

When number of moles is given, we can calculate the amount of heat released by multiplying $\Delta H_f^{\circ}$ by the number of moles since $\Delta H_f^{\circ}$ is for exactly 1 mole.

\subsection*{Example 4}
Calculate the amount of heat released when $3.76mol$ of
$Ca(OH)_2$ are allowed to react.

\begin{align*}
    R: Ca(OH)_2+CO_2\rightarrow CaCO_3+H_2O|\Delta H=-69.1 kJ(nice!)
\end{align*}

\begin{align*}
    \Delta H_r=-69.1(3.76)\approx -259.8kJ
\end{align*}

\subsection*{Calorimetry with constant volume (applications)}

We obtain the heat exchange of a reaction indirectly through means of a calorimeter.

\begin{align*}
    q_{rxn}=-q_{calorimeter}
\end{align*}

\begin{itemize}
    \item In a constant volume scenario (bomb calorimetry), we have that $q_v=\Delta U$
    \item In a constant pressure scenario, we have that $q_P=\Delta U+P\Delta V$
\end{itemize}

Consider a system $\Sigma$ where $\Sigma$ contains reactants in the calorimeter, an inner water bath, and a calorimeter vessel (by itself without the water in it). We can model this system's internal energy with the following equation::

\begin{align*}
    \Delta U_{\Sigma}=\frac{m_S}{M_S}\Delta_C U+\frac{m_{H_2O}}{M_{H_2O}}C_{P,m}(H_2O,l)\Delta T+C_{cal}\Delta T=0
\end{align*}

We have to know the following to use this model::

\begin{itemize}
    \item mass and molar mass of sample
    \item mass and heat capacity of water
    \item heat capacity of calorimeter (determined via separate experiment)
\end{itemize}

We also wish to measure $\Delta T$ whereby we can measure $\Delta U$.

\subsection*{Example}
The dipeptide glycylglycine ($C_4H_8O_3N_2$) has a standard enthalpy of combustion of $-1969 kJ mol^{-1}$. Calculate $q$, $w$, $\Delta U$, and $\Delta H$ when $10.0 g$ of glycine are burned at $T = 298 K$ and a constant pressure of $1.00 bar$. Assume the
combustion products are \textbf{carbon dioxide gas, nitrogen gas, and liquid water}.

Our reaction is the following::

\begin{align*}
    R: C_4H_8O_3N_2(s)+\frac{9}{2}O_2(g)\rightarrow N_2(g)+4CO_2(g)+4H_2O(l), n(C_4H_8O_3N_2)=0.0757
\end{align*}


\begin{align*}
    \Delta H_{comb}=_{comb}=\frac{(-1969)(10)}{132.12}=-149.0kJ
\end{align*}

\begin{align*}
    \Delta U_{comb}=\Delta H{comb}-\Delta nRT=-149.12kJ
\end{align*}

\begin{align*}
    w_{comb}=-p_{ext}\Delta V=-RT\Delta n=-93.78J
\end{align*}


\subsection*{Constant presure calorimetry}

For an isobaric system $\Sigma$, we are mainly concerned about the change in enthalpy of the system.

\begin{align*}
    \Delta H_\Sigma^{\circ}=\frac{m_S}{M_s}\Delta_s H^{\circ}+\frac{m_{H_2O}}{M_{H_2O}}C_{P,m}(H_2O,l)\Delta T+C_{cal}\Delta T=0
\end{align*}

\subsection*{Example}

In a constant pressure calorimeter, $33.3 mL$ of $0.100 M$ $AgNO_3$ is mixed with $33.3 mL$ of $0.100 M HCl$. The
reaction $Ag^+(aq) + Cl^-(aq)$ takes place. The temperature of the reactants was $25.60 ^{\circ}C$ and the final temperature is
$26.47^{\circ}C$. Calculate the molar enthalpy change for this reaction.\\

First, we can calculate the moles of $AgNO_3$

\begin{align*}
n(AgNO_3)=n(HCl)=(0.0333L)(0.100M)=0.0033 mol
\end{align*}

\begin{align*}
    q=m_S C_S\Delta T=2(33.3)(4.18)(26.47-25.6)=242.2J\implies q_{rxn}=-242.2J
\end{align*}

So for molar enthalpy, we simply divide by number of moles.

\begin{align*}
    \Delta H_{rxn}=\frac{-242.2}{0.00333}=-72.7\frac{kJ}{mol}
\end{align*}

\subsection*{Differential scanning calorimetry}

It's just bio stuff. Don't worry about it.

\chapter*{Practice problems (tutorial)}

\subsection*{Question 1}

Benzoic acid ($C_6H_5COOH$), $1.35 g$, is reacted with oxygen in a constant volume calorimeter to form
$H_2O(l)$ and $CO_2(g)$. The mass of the water in the inner bath is $1.240\times 10^3 g$. The temperature of the
calorimeter and its content rise $3.45 K$ as a result of this reaction. Calculate the calorimeter constant.
Note that the standard enthalpy of combustion of benzoic acid ($298 K$) is $-3228.0 kJ mol-1$

First, we need to balance our equation.

\begin{align*}
    2C_6H_5COOH(aq)+\frac{15}{2}O_2(g)\rightarrow 7CO_2(g)+3H_2O(l)
\end{align*}

Because $V\equiv const$, we have that $\Delta U=q_{rxn}$. To find $\Delta U$, we need to use $\Delta H$.

\begin{align*}
    \Delta U=\Delta H-\Delta nRT=-3228.0-(7-7.5)\left(\frac{8.314}{1000}\right)(298K)=-3226.8 \frac{kJ}{mol}
\end{align*}

So to calculate the calorimeter constant, 

\begin{align*}
    q_{rxn}=n\Delta U_{rxn}=-35.72kJ=-q_{surr}
\end{align*}

\begin{align*}
    q_{surr}=q_w+q_{cal}=m_w C_w \Delta T+C_{cal}\Delta T=35720J
\end{align*}

After plugging in our values and isolating for $C_{cal}$, we find that $C_{cal}=-322.8\frac{kJ}{mol}$
\newpage
\subsection*{Question 2}

Under anaerobic conditions, glucose is broken down in muscle tissue to form lactic acid according to the reaction $C_6H_{12}O_6\rightarrow 2CH_3CHOHCOOH$.
The standard enthalpy of formation ($T = 298 K$) for glucose and lactic acid are $-1273.1 kJ mol^{-1}$ and $-673.6 kJ mol^{-1}$;
and the $C_P$ are $219.2 J/K mol$ and $127.6 J/k mol$, respectively. Calculate the enthalpy of reaction when $5 g$ of
glucose react at $330 K$ assuming the change in heat capacity of the reaction is negligible.

\begin{align*}
    \Delta H_r=2(-673.6)-(-1273.1)=-74.1\frac{kJ}{mol}|T=298K
\end{align*}

We need $\Delta C_P$ to ind $\Delta H_r$ at $330K$.

\begin{align*}
    \Delta C_P(2(127.6)-219.2)=36\frac{J}{Kmol}
\end{align*}

So,

\begin{align*}
    \Delta H_{320K}=\Delta H_{298K}+\int_{298}^{330}C_P dt=74100+(36)(320-298)=-72948\frac{J}{mol}
\end{align*}

Scaling to 5 grams, we have that

\begin{align*}
    \frac{5}{180}(-72948)=2.03kJ
\end{align*}

\subsection*{Question 4}
A sample of $K(s)$ of mass $3.041 g$ undergoes combustion in a constant volume calorimeter to
produce $K_2O(s)$.\\

The calorimeter constant is $1849 JK^{-1}$, and the measured temperature rise in the inner water
bath containing $1450 g$ of water is $1.776 K$. $C_{water} = 4.184 J g^{-1} K^{-1}$
Calculate $\Delta_f H$ and $\Delta_f U$ for $K_2O$.\\

We can start by balancing the equation per mole.

\begin{align*}
    R: 2K(s)+\frac{1}{2}O_2(g)\rightarrow K_2O(s)
\end{align*}
So now we can calculate our values.
\begin{align*}
    q_{rxn}=-q_{surr}=\Delta U
\end{align*}

\begin{align*}
    q_{surr}=m_{water}C_{water}\Delta T+C_{cal} \Delta T=(1450)(4.184)(1.776)+(1849)(1.776)=14058J
\end{align*}

To find $\Delta_f U$ and $\Delta_f H$, we need number of moles.

\begin{align*}
    n(K_2O)=\frac{1}{2}n(K)=\frac{3.041}{2(39.098)}=0.0389 mol
\end{align*}

\begin{align*}
    \Delta_f U=\frac{q_{rxn}}{n(K_2O)}=\frac{-14058}{0.0389}=-361.4\frac{kJ}{mol}
\end{align*}

\begin{align*}
    \Delta_f H=\Delta_f U+()\Delta n) RT=-361400+\left(\frac{-1}{2}\right)(8.314)(298)=362.7 \frac{kJ}{mol}
\end{align*}

\end{document}