\documentclass[12pt]{article}

\usepackage[]{amsmath}
\usepackage[]{amsthm}
\usepackage[]{amsfonts}
\usepackage[]{amssymb}
\usepackage{blindtext}
\usepackage{pgfplots}
\usepackage[a4paper, total={6in, 8in}]{geometry}

\title{Chapter 2:: The first law of thermodynamics}

\author{Jerry Wu (217545898)}

\date{\today}
\begin{document}

\maketitle
\rule{\textwidth}{0.4pt}
Essentially, the first law states that energy is always conserved.\\

\subsection*{Internal energy}

In this chapter, we'll define energy as the capacity to supply heat or do work. A system doesn't inherently have heat or work; only if it's transferred. We measure it in the following units::

\begin{align*}
    1J=\frac{1kgm^2}{s^2}
\end{align*}

\begin{align*}
    1calorie=4.184J
\end{align*}
\begin{align*}
    1calorie_{dietary}=1000calories
\end{align*}

Energy cannot be created or destroyed. Only transferred or converted from one form/object to another.\\

\begin{itemize}
    \item \textbf{Energy} is the capacity to do work
    \item In thermodynamics, we are interested in flow of energy in \textbf{heat} and \textbf{work}. This is because they describe the process of energy transfer.
\end{itemize}

In essence, for any system $\Omega$,\\
\begin{align*}
    \sum_{\Omega}{E}=E_{kinetic} + E_{potential}
\end{align*}

\subsection*{Forms of energy}
\begin{itemize}
    \item Thermal energy (kinetic and heat)
    \item Bond energy (potential)
\end{itemize}

\subsection*{The first law}

Because energy can't be created or destroyed, we can express change in energy in the following ways::

\begin{align*}
    \Delta E_{\Omega}=\Delta E_{\sigma}+\Delta E_{\sigma^c}=0
\end{align*}

\begin{align*}
    \implies\Delta E_{\sigma}=-\Delta E_{\sigma^c}
\end{align*}

$\forall\Omega$ at rest with no external fields, it can be said that

\begin{align*}
    E_{\sigma}=
    \Delta U_{\Omega}=\Delta U_{\sigma}+\Delta U_{\sigma^c}
\end{align*}

$U$ (internal energy) is a state function, so \begin{align*}\Delta U=U_2-U_2=q+w\end{align*}

In the simplest processes, only one of $P$, $V$, or $T$ will change at any given time. Some important things to note are the following::

\begin{itemize}
    \item Isothermal: const temperature, $(\Delta T_\sigma=0)$
    \item Isobaric: const pressure, $(\Delta P_\sigma=0)$
    \item Isochoric: const volume, $(\Delta V_\sigma=0)$
    \item Adiabatic no heat transfer $(q=0)$
    \item Cyclic $\implies \Sigma_1=\Sigma_0$, where $\Sigma_n$ is some arbitrary state or property of a system.
\end{itemize}

\subsection*{Internal energy vs heat/work}
In a reaction, change in internal energy is expressed as::
\begin{align*}
    \Delta U_{\Omega}=\Delta U_{products}-\Delta U_{reactants}
\end{align*}

Recall that work is \textbf{NOT} a state function!
\begin{itemize}
    \item When work is positive, $\Delta U_\sigma>0$ and $\Delta V_\sigma<0$
    \item When work is negative, $\Delta U_\sigma<0$ and $\Delta V_\sigma>0$
\end{itemize}

\subsection*{Expansion work}
Let us start by looking at the two pressure inequalities below::
\begin{align*}
    P_\sigma > P_{\sigma^c} \implies expansion\\
    P_\sigma < P_{\sigma^c} \implies compression
\end{align*}

So we have that

\begin{align*}
    P_{\sigma^c} = \frac{F_{\sigma}}{A}\\
    dV = Adh
\end{align*}

Then

\begin{align*}
    dw = -F_{\sigma^c} dh = -\frac{F_{\sigma^c}}{A}Adh = -P_{\sigma^c}dV
\end{align*}
    
Thus, our final formula for expansion work can be expressed as the following integral::

\begin{align*}
    w=-\int_{v_1}^{v_2}P_{\sigma^c}dV
\end{align*}

where $P_{\sigma^c}$ is a function of the volume of gas in an arbitrary container $\sigma$.

\subsection*{Why do we care about work?}
We care because energy is transferred from the system to the surroundings in the form of work (only applicable to adiabatic systems).

\subsection*{Work with gases}
We can express this idea with the equation:

\begin{align*}
    w=-\Delta nRT, \Delta n=v_{prod,(g)}-v_{rxt,(g)}
\end{align*}

Take the following reaction for example::
\begin{align*}
    C_3H_8(g)+5O_2(g)\implies3CO_2(g)+4H_2O(g)
\end{align*}
Here, we have 6 moles of gas on the reactant side, and 7 on the product side. So we have that $\Delta n=1$. This means that work was done by the system, thus the sign for work is negative.\\In general::
\begin{align*}
    \Delta n>0\implies w<0\\
    \Delta n<0\implies w>0
\end{align*}

\subsection*{Example (slide 18)}
A greatly simplified model of an internal combustion engine is as follows. At the
start of the power stroke the ignited gases exert a pressure of 20 atm, and drive the
piston back against a constant force equivalent to 5.0 atm. In so doing, the piston
sweeps out 250 cm3. What is the power output of a six-cylinder engine working at
2000 RPM (with one power stroke from each cylinder every second revolution)?

\begin{itemize}
    \item[\textbf{Solution::}] We can start by calculating work.
    \begin{align*}
        w=-P\Delta V=-(5atm)(0.25L)=1.25Latm=1.25Latm\frac{101.325J}{Latm}=1.3E2J
    \end{align*}

    In six cylinders over a 1 minute duration, we have that,
    \begin{align*}
        (2000)(6)(w)(0.5)=7.8E5J
    \end{align*}

    Rate of production::

    \begin{align*}
        P=\frac{w}{t}=\frac{7.8E5}{60s}=13LW
    \end{align*}
\end{itemize}

\subsection*{Example 2 (slide 19)}
\begin{align*}
    R:Zn(s)+2H^+(aq)+2Cl^-(aq)\implies H_2(g)+Zn^{2+}(aq)+2Cl^-(aq)
\end{align*}
\begin{itemize}
    \item[a)]  How much work has been done if 5 g of Zn have completely reacted according
    to the following reaction if the external pressure is 1 atm and T=298K.

    \begin{align*}
        n(H_2)=n(Zn)=\frac{5g}{65.38\frac{g}{mol}}=0.0765mol
    \end{align*}
    \begin{align*}
        \Delta V=\frac{n(H_2)RT}{P}=\frac{0.0765mol(0.08206\frac{Latm}{Kmol})298K}{1atm}=1.87L
    \end{align*}

    \item[b)] What’s the work per 1 mole of Zn consumed?
    
    \begin{align*}
        w=-P\Delta V=\frac{-189J}{0.0765mol}=-2470\frac{J}{mol}
    \end{align*}
\end{itemize}

\subsection*{Heat}
Recall that heat is not a state function. If $\Delta Q=0$, the system is at thermal equilibrium. Conservation of energy requires that $q_{\sigma}+q_{\sigma^c}=0\implies q_{\sigma}=-q_{\sigma^c}$.

\subsection*{Example 3 (slide 26)}

How much heat is transferred in raising the temperature of a beaker of water (50 g) from 25°C to 35°C (at constant P = 1 atm)?

\begin{align*}
    q=mc\Delta T=(50g)(4.1840\frac{J}{G^{\circ}})(10^{\circ})=2092J
\end{align*}

\subsection*{Heat capacity}

Assume $\sigma$ is closed with no phase changes or reactions.

\begin{align*}
    q_\sigma=\int_{T_1}^{T_2}CdT=\int_{T_1}^{T_2}mC^c dt
\end{align*}

Temperature rising depends on the following::
\begin{itemize}
    \item[1.] Amount of heat delivered
    \item[2.] Amount of substance present
    \item[3.] Chemical nature and physical state of substance
    \item[4.] Conditions under which heat is added to substance
\end{itemize}
Specific heat capacities can be found on \textbf{slide 30}. Recall that the higher the heat capacity, the harder it is to add temperature to the substance.\\

At constant volume::
\begin{align*}
    C_V=\frac{\Delta U}{\Delta T}=\frac{q_v}{\Delta T}
\end{align*}

At constant pressure::
\begin{align*}
    C_P=\frac{\Delta H}{\Delta T}=\frac{q_v}{\Delta T}
\end{align*}

Heat capacity can also be expressed as the limit::
\begin{align*}
    \lim_{\Delta T \rightarrow 0}\frac{q}{\Delta T}
\end{align*}

$\Delta U=f(T)$. Internal energy is not a function of volume or pressure. The change in enternal energy can be expressed as the partial differential::

\begin{align*}
    \frac{\partial U}{\partial V}_T=0 \forall g\in\mathbb{C} 
\end{align*}

For sake of simplicity, we'll assume all processes covered in this course are quasi-static.\\

Internal equilibrium $\equiv$ systems are homogenous.

\subsection*{Reversible \& irreversible processes}

\begin{itemize}
    \item [\textbf{Reversible::}] An infinitessimal change \textbf{will} change the direction of the process
    \item [\textbf{Irreversible::}] An infinitessimal change \textbf{will not} change the direction of the process
\end{itemize}

In reality, there's no process that's fully reversible. They just progress much slower than irreversible processes.

\subsection*{PV diagrams (slide 43, 44)}
Take a function $V=f(P)$. We want to study the region bound between $P_1$ and $P_2$ as well as $V_1$ and $V_2$. Isothermal curves are hyperbola, because the function is the rational function::
\begin{align*}
    PV=nRT\implies P=f(V)=\frac{nRT}{V}
\end{align*}

Where n and T are constant.

\subsection*{Example 1}

We have 3 moles of an arbitrary ideal gas expanding isothermally ($T=const$). $V_i=20.0L$, $V_1=60.0L$, $T=27.0^{\circ}C$. Calculate the work in this process.

\subsection*{Solution}
\begin{align*}
    w=-P\Delta V=-(1.0 atm)(60-20)L=-40L atm=-40Latm(\frac{101.35J}{Latm}) =-4053J=-4.1kJ
\end{align*}

\subsection*{Infinitessimal pressure changes for expansion}
When this occurs, we need to use integration. Recall that

\begin{align*}
    w=-\int_{V_1}^{V_2}P_{ex}dV=-\int_{V_1}^{V_2}(P-dP)dV=-\int_{V_1}^{V_2}PdV
\end{align*}

Because $P=\frac{nRT}{V}=const$, we have that

\begin{align*}
    -nRT\int_{V_1}^{V_2}\frac{1}{V}dV=-nRT(\ln(V_2)-\ln(V_2))=-nRT(\ln(\frac{V_2}{V_1}))
\end{align*}

By laws of logarithms.\\Thus, we have our derived formula::
\begin{align*}
    w=-nRT(\ln(\frac{V_2}{V_1})) \blacksquare
\end{align*}

\textbf{NOTE::} For any real process $\rho\in\mathbb{R}$, $\rho$ is irreversible.\\

\subsection*{Summary}

\begin{itemize}
    \item For expansion, $|w_{irreversible}|\geq |w_{reversible}|$
    \item For compression, $|w_{irreversible}|\leq |w_{reversible}|$
    \item For irreversible, $w=-P_{ex}\Delta V$
    \item For reversible, $w=-nRT(\ln(\frac{V_2}{V_1})$
\end{itemize}

\subsection*{Example 1 revisited}
We have 3 moles of an arbitrary ideal gas expanding isothermally ($T=const$). $V_i=20.0L$, $V_1=60.0L$, $T=27.0^{\circ}C$. Calculate the work in this process.

\begin{align*}
    w=-nRT(\ln(\frac{V_2}{V_1}))=-(3mol)(8.314\frac{J}{Kmol})(300K)(\ln(\frac{60L}{20L}))=-8.22kJ
\end{align*}

We get a greater resultant work by assuming the process is reversible.

\subsection*{First law of thermodynamics summary}
Recall that in a temperature dependent system, heat is calculated as::
\begin{align*}
    q=\int_{T_1}^{T_2}CdT=\int_{T_1}^{T_2}m\bar{c}dT
\end{align*}

More formulas on slide 58 and 59.

\subsection*{Enthalpy revisited}

\begin{align*}
    H=U+PV
\end{align*}

$U$, $P$, and $V$ are all state functions $\implies$ $H$ is also a state function. However, we don't care about absolute enthalpy. We care about change in enthalpy. So,

\begin{align*}
    \Delta H=\Delta U+\Delta (PV)=\Delta U+P\Delta V+V\Delta P+\Delta P\Delta V
\end{align*}

Assume $P\equiv const$, we have that,

\begin{align*}
    \Delta H=\Delta U+P\Delta V=\Delta U-w=q_p
\end{align*}

For $g\in \mathbb{C}$,
\begin{align*}
    \Delta H=C_P\Delta T
\end{align*}

For infinitessimal change in $T$,
\begin{align*}
    dH=dU+d(nRT)=dU+nRdT
\end{align*}

Knowing that $dH=C_P dT$ and $dU=C_V dT$, we can substitute into the above equation:
\begin{align*}
    C_P dT=C_V dT+nRdT\\
    C_P=C_V+nR\\
    C_P-C_V=nR\\
    \bar{C_P}-\bar{C_V}=R
\end{align*}

\subsection*{Difference between $\Delta H$ and $\Delta U$}

Consider the following reaction::

\begin{align*}
    R:2Na_{(s)}+2H_2O_{(l)}\implies 2NaOH_{(aq)}+H_{2(g)}\\
    \Delta H=-367.5kJ
\end{align*}

Volume of $H_2$ generated at 1 atm is $24.5L$, so $-P\Delta V=-24.5Latm=-2.5kJ$, $\Delta U=-370kJ$. We can see that $\Delta H\approx\Delta U$


\subsection*{Example with heat capacity (slide 64)}
Calculate $\Delta H$ and $\Delta U$ for the transformation of 1.00 mol of an ideal gas from $27.0^{\circ}C$
and 1.00 atm to 327°C and 17.0 atm if

\begin{align*}
    C_{p,m}=20.9+0.42\frac{T}{K} JK^{-1}mol^{-1}
\end{align*}

For an ideal gas, $\Delta H$ is given by:

\begin{align*}
    \Delta H=\int_{T_1}^{T_2}C_P dT=\int_{T_1}^{T_2}C_{p,m}ndT=\int_{300}^{600}(20.9+0.042\frac{T}{K})dT\\=(20.9T+\frac{0.042T^2}{2})\vert_{300}^{600}=(20.9(600-300)+0.021(600^2-300^2))\\
    \approx 11.9\times 10^3J=11.9kJ
\end{align*}

\subsection*{Adiabatic expansion}
For adiabatic processes, we always assume that $\Omega\equiv adiabatic\implies q=0\forall\Omega$, so,

\begin{align*}
    dU=dw=-PdV=-\frac{nRT}{V}dV\\
    \frac{dU}{nT}=-R\frac{dV}{V}
\end{align*}
since
\begin{align*}
    dU=C_V dt
\end{align*}

So,

\begin{align*}
    \frac{dU}{nt}=\frac{C_V dt}{nT}=\bar{C_V}\frac{dT}{T}=-R\frac{dV}{V}
\end{align*}

Integrating both sides, we have that
\begin{align*}
    \int_{T_1}^{T_2}\bar{C_V}\frac{dT}{T}=-R\int_{T_1}^{T_2}\frac{dV}{V}=\bar{C_V}\ln(\frac{T_2}{T_2})=R\ln(\frac{V_1}{V_2})
\end{align*}

Since $\bar{C_P}-\bar{C_V}=R$ for an ideal gas, then.
\begin{align*}
    \bar{C_V}\ln(\frac{T_2}{T_1}=)\bar{C_P}-\bar{C_V}\ln(\frac{V_1}{V_2})
\end{align*}

Divide both sides by $\bar{C_V}$ to obtain::

\begin{align*}
    \ln(\frac{T_2}{T_1})=(\frac{\bar{C_P}}{\bar{C_V}}-1)\ln(\frac{V_1}{V_2})=(\gamma-1)\ln(\frac{V_1}{V_2})=\ln(\frac{V_1}{V_2})^{\gamma-1}, \gamma=\frac{\bar{C_P}}{\bar{C_V}}
\end{align*}

\textbf{$\bar{C_P}$ and $\bar{C_V}$ constants wont be provided on the constant sheet! Review chapter 2 slide 71 for them!}

\subsection*{Reversible adiabatic expansion ($g\in \mathbb{C}$)}

\begin{align*}
    T_1V_1^{\gamma-1}= T_2V_2^{\gamma-1}=
\end{align*}

We can also do this for $T$ and $P$

\begin{align*}
    P_1^{1-\gamma}T_1^{\gamma}=P_2^{1-\gamma}T_2^{\gamma}
\end{align*}

\begin{align*}
    T_1P_1^{\frac{1-\gamma}{\gamma}}=T_2P_2^{\frac{1-\gamma}{\gamma}}
\end{align*}

\subsection*{Work done in an adiabatic expansion (slide 74-79)}

3.50 moles of an ideal gas are expanded from 450. K and an initial pressure of 5.00 bar to a
final pressure of 1.00 bar, and CP,m = 5/2R. Calculate w for the following two cases:

\begin{itemize}
    \item[a)] The expansion is isothermal and reversible.
    \item[b)] The expansion is adiabatic and reversible.


\end{itemize}

$w$ for an isothermal process is given by::

\begin{align*}
    w=-nRT(\ln(\frac{V_2}{V_1}))
\end{align*}

Calculate $V_0$ and $V_f$. Recall that $PV=nRT$ always holds!

\begin{align*}
    V_1=\frac{nRT_1}{V_1}=\frac{(13.5)(0.08314)(450K)}{(5.00)}=26.2L
\end{align*}

\begin{align*}
    V_2=\frac{nRT_2}{V_2}=\frac{(3.5)(8.314)(450)}{10^5}=0.1310m^3
\end{align*}

\begin{align*}
    w=-nRT(\ln(\frac{V_2}{V_1}))=(-3.5)(8.314472)(450K)(\ln(\frac{0.1310}{0.0262}))=-21076J=-21.1kJ
\end{align*}

For adiabatic:

\begin{align*}
    C_V=C_P-R=\frac{5}{2}R-R=\frac{3}{2}R
\end{align*}

\begin{align*}
    T_1P_1^{\frac{1-\gamma}{\gamma}}=T_2P_2^{\frac{1-\gamma}{\gamma}}
\end{align*}

Now we find $T_2$

\begin{align*}
    T_2=\frac{T_1P_1^{\frac
    1-\gamma}{\gamma}}{P_2^{\frac{1-\gamma}{\gamma}}}
\end{align*}

After some plugging in, we get $T_2=236K$. Now we can calculate work!

\begin{align*}
    w=C_V(T_2-T_1)=\frac{3}{2}(8.314)(3.5)(236-450)=-9.23kJ
\end{align*}

In general, gases with the same parameters will have the following property::

\begin{align*}
    |w|_{isothermal(exp)}>|w|_{adiabatic(exp)}
\end{align*}

\subsection*{Irreversible adiabatic expansions}
Assume $P_2=P_{ext}$ and $q=0$ by definition of adiabatic.

\begin{align*}
    \Delta U=w
\end{align*}
\begin{align*}
    n\bar{C}_V(T_2-T_1)=-P_2(V_2-V_1)
\end{align*}
But,
\begin{align*}
    V_1=\frac{nRT_1}{P_1}, V_2=\frac{nRT_2}{P_2}
\end{align*}
So,
\begin{align*}
    n\bar{C}_V(T_2-T_1)=-P_2(\frac{nRT_2}{P_2}-\frac{nRT_1}{P_1})
\end{align*}

\subsection*{Example}

An automobile tire contains air at $320\times10^3Pa$ at $20.0^{\circ}C$. The stem valve is
removed and the air is allowed to expand adiabatically against the constant
external pressure of $100\times10^3 Pa$ until $P=P_{external}$. For air, $C_{V,m}=\frac{5}{2}R$. Calculate
the final temperature. Assume ideal gas behavior.

Since $q=0$, we have that $\Delta U=w$.

\begin{align*}
    C_{V,m}n(T_2-T_1)=-P_{ext}(V_2-V_1)=-P_2(\frac{nRT_2}{P_2}-\frac{nRT_1}{P_1})
\end{align*}

We can cancel the $n$ values, (full solution on \textbf{slide 80}). It's all plugging in. Our final solution would be $0.804$.

\end{document}