\documentclass[12pt]{book}

\usepackage[]{amsmath}
\usepackage[]{amsthm}
\usepackage[]{amsfonts}
\usepackage[]{amssymb}
\usepackage{blindtext}
\usepackage{pgfplots}
\usepackage[a4paper, total={6in, 8in}]{geometry}

\title{CHEM2011 office hours notes}
\author{Jerry Wu}
\date{2023-02-22}

\begin{document}
\maketitle
\chapter*{Test 1 notes}

\subsection*{Terminology and cases}

\begin{itemize}
    \item \textbf{Isothermal} $\implies T\equiv const\implies w=-nRT\ln(\frac{V_2}{V_1})=-nRT\ln(\frac{P_1}{P_2})$
    \item \textbf{Isobaric} $\implies P\equiv const\implies w=-P\Delta V$
    \item \textbf{Isochoric} $\implies V\equiv const\implies w=0,\Delta U=q$
    \item \textbf{Adiabatic} $\implies q=0\implies \Delta U=w\implies T_1P_1^{\frac{1-\gamma}{\gamma}}=T_2P_2^{\frac{1-\gamma}{\gamma}}$
\end{itemize}

\subsection*{Definition of a state function and constants}

\begin{itemize}
    \item A function $f(x,y)$ is a state function if and only if the following property is satisfied::
    \begin{align*}
        \left(\frac{\left(\frac{\partial f}{\partial x}\right)_y}{\partial y}\right)_x=\left(\frac{\left(\frac{\partial f}{\partial x}\right)_y}{\partial y}\right)_x
    \end{align*}
    \item For monoatomic gases, $\bar{C_V}=\frac{3}{2}R$, $\bar{C_P}=\frac{5}{2}R$, $\gamma=\frac{\bar{C_P}}{C_V}=\frac{5}{3}$
    \item For diatomic gases (assume no vib.),$\bar{C_V}=\frac{5}{2}R$, $\bar{C_P}=\frac{7}{2}R$, $\gamma=\frac{\bar{C_P}}{C_V}=\frac{7}{5}$
\end{itemize}



\end{document}